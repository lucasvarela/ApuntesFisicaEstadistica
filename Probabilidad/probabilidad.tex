\documentclass[oneside]{book}
\usepackage{braket}
\usepackage[latin1]{inputenc}
\usepackage{amsfonts}
\usepackage{amsthm}
\usepackage{amsmath}
\usepackage{mathrsfs}
\usepackage{empheq}
\usepackage{enumitem}
\usepackage[pdftex]{color,graphicx}
\usepackage{hyperref}
\usepackage{listings}
\usepackage{calligra}
\usepackage{algpseudocode} 
\DeclareFontShape{T1}{calligra}{m}{n}{<->s*[2.2]callig15}{}
\newcommand{\scripty}[1]{\ensuremath{\mathcalligra{#1}}}
\setlength{\oddsidemargin}{0cm}
\setlength{\textwidth}{490pt}
\setlength{\topmargin}{-40pt}
\addtolength{\hoffset}{-0.3cm}
\addtolength{\textheight}{4cm}
\usepackage{amssymb}
\usepackage{graphicx} % Required for the inclusion of images
\setlength\parindent{0pt} % Removes all indentation from paragraphs
\usepackage{float}
\usepackage{makeidx}

%\begin{figure}[H]
%	\centering
%	\includegraphics[scale = 0.42]{lcaoderecha}
%	\caption{Eletr\'on ligado solo al n�cleo derecho}
%	\label{fig1}
%	\end{figure}




\begin{document}
%\tableofcontents
%\pagebreak










\begin{center}
\textsc{\LARGE F�sica Estad�stica}\\[0.5cm]
\textsc{\LARGE Probabilidad}\\[0.5cm]
\textsc{\LARGE Soluci\'on}\\[0.5cm]
\end{center}


\begin{center}
\begin{tabular}{l r}
\large
Notas de Clase de los Profesores: Alonso Botero \& Gabriel T�llez% Instructor/supervisor
\normalsize
\end{tabular}
\end{center} %\\[0.5cm]



\textbf{\large Estas son las notas de clase tomadas en los semestres 2014-2 \& 2015-1 en la clase F�sica Estad�stica dictadas por los profesores Gabriel T�llez \& Alonso Botero respectivamente. Estas notas son escritas por un alumno y pueden contener errores, uselas con precauci\'on.}\\


\tableofcontents
\pagebreak

\chapter{ Introducci�n}

Las probabilidades hacen una parte fundamental de la f�sica estad�stica. Ac� se har� una breve introducci�n a ella para poder utilizar estas nociones en la descripci�n de sistemas f�sicos. Primero se comienza por introducir la notaci�n de probabilidad (\ref{1.1}).


\begin{equation}
\label{1.1}  P(A|B)  \equiv  \textbf{La probabilidad de que ocurra "A" dado que ya paso "B"}
\end{equation}

La probabilidad $P(A|B)$ se define de tal forma que si $P(A|B)=1$ siempre va a ocurrir el evento $A$ dado $B$ y por el contrario si $P(A|B)=0$, nunca va a ocurrir $A$ dado $B$. Adem�s se introduce la notaci�n (\ref{1.2} - \ref{1.4} ):



\begin{align}
\label{1.2}  A B \equiv \textbf{A o B} \\
\label{1.3} \bar{A} \equiv \textbf{No A} \\
\label{1.4} P(A) \equiv \textbf{Se sobre entiende la condici�n}
\end{align}

Ser�n �tiles tambi�n las leyes de Morgan (\ref{1.5} - \ref{1.6} ):


\begin{align}
\label{1.5}  \overline{AB} =  \bar{A} + \bar{B} \\
\label{1.6} \overline{A+B} =  \overline{AB}
\end{align}


\chapter{Reglas B�sicas de Probabilidad}

\end{document}